
\documentclass[10pt]{article}
\usepackage[utf8]{inputenc}
\usepackage[brazil]{babel}
\usepackage{amsmath}
\usepackage{graphicx}
\usepackage{hyperref}
\usepackage{enumitem}
\usepackage[a4paper,margin=2.5cm]{geometry}

\title{Uma Análise Comparativa dos Tipos de Conhecimento: Popular, Científico, Filosófico e Religioso}
\author{Seu Nome \\ Sua Instituição \\ \texttt{seu.email@exemplo.com}}
\date{}

\begin{document}

\maketitle

\begin{abstract}
Este artigo tem como objetivo apresentar e comparar os quatro principais tipos de conhecimento segundo Marconi e Lakatos (2022): popular, científico, filosófico e religioso. A análise considera suas características, métodos de aquisição, fundamentos e áreas de aplicação. Também se utiliza como fonte complementar a obra de Chalmers (1999), que discute o conhecimento científico a partir de uma perspectiva epistemológica. O artigo busca fornecer uma compreensão clara das distinções e inter-relações entre esses conhecimentos, destacando suas contribuições para a formação do pensamento humano.
\end{abstract}

\textbf{Palavras-chave:} Conhecimento, Epistemologia, Filosofia, Ciência, Religião

\section{Introdução}
O conhecimento, em suas diversas formas, sempre desempenhou papel central na história da humanidade. Seja para interpretar fenômenos naturais, conduzir descobertas científicas, desenvolver reflexões filosóficas ou sustentar crenças religiosas, diferentes tipos de conhecimento se estabeleceram como formas válidas --- ainda que distintas --- de compreender o mundo. Conforme Marconi e Lakatos (2022), há quatro formas principais de conhecimento: o conhecimento popular, o científico, o filosófico e o religioso.

\section{Tipos de Conhecimento}

\subsection{Conhecimento Popular}
O conhecimento popular, ou empírico, é adquirido de forma espontânea, sem método sistemático. Baseia-se na experiência cotidiana, em observações diretas e em tradições orais ou culturais. É utilitário e frequentemente acrítico, sendo transmitido de geração em geração.

\subsection{Conhecimento Científico}
O conhecimento científico distingue-se pelo seu caráter sistemático, racional e verificável. Ele é obtido por meio da observação rigorosa, da formulação de hipóteses e da testagem empírica. Segundo Chalmers (1999), a ciência não é um reflexo direto da realidade, mas uma construção baseada em modelos teóricos.

\subsection{Conhecimento Filosófico}
O conhecimento filosófico é caracterizado pela racionalidade e pelo uso intensivo da reflexão e da argumentação lógica. Ele busca respostas para questões fundamentais da existência, do ser, do conhecimento e dos valores.

\subsection{Conhecimento Religioso}
O conhecimento religioso é baseado na fé e na revelação divina. Ele é aceito como verdade absoluta por seus adeptos, independentemente de comprovação empírica ou racional.

\section{Comparação entre os Tipos de Conhecimento}
A Tabela~\ref{tab:comparacao} apresenta uma visão geral comparativa entre os quatro tipos de conhecimento.

\begin{table}[h!]
\centering
\begin{tabular}{|l|l|l|l|l|}
\hline
\textbf{Tipo} & \textbf{Aquisição} & \textbf{Fundamento} & \textbf{Caráter} & \textbf{Exemplo} \\ \hline
Popular & Experiência & Observação/tradição & Acrítico, prático & Chá de camomila acalma \\ \hline
Científico & Método científico & Observação/lógica & Sistemático, verificável & Água ferve a 100ºC \\ \hline
Filosófico & Racional-reflexivo & Argumentação lógica & Abstrato, crítico & O que é a verdade? \\ \hline
Religioso & Fé e revelação & Textos/dogmas & Absoluto, emocional & Deus criou o mundo \\ \hline
\end{tabular}
\caption{Comparação entre os tipos de conhecimento}
\label{tab:comparacao}
\end{table}

\section{Considerações Finais}
Compreender os diferentes tipos de conhecimento é essencial para reconhecer a pluralidade de formas pelas quais os seres humanos constroem sentido sobre a realidade. Embora distintos em método e fundamento, esses conhecimentos não são excludentes, mas complementares. Saber diferenciá-los contribui para o exercício do pensamento crítico e para o diálogo entre ciência, fé, razão e experiência.

\begin{thebibliography}{9}
\bibitem{chalmers}
CHALMERS, A. F. \textit{O que é ciência afinal?} 3. ed. São Paulo: Brasiliense, 1999.

\bibitem{marconi}
MARCONI, M. A.; LAKATOS, E. M. \textit{Fundamentos de metodologia científica}. 9. ed. São Paulo: Atlas, 2022. Cap. 1 e 5.

\end{thebibliography}

\end{document}
